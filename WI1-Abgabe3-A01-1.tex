% Createt 2016 by Jens Wiebe for the use in the course: Wirtschaftinformatik 1 at Universität Bremen
% The use of TexLive 2016 as compiler-package and TexStudio 2.11.2 as editor are recommended

\documentclass[12pt,utf8]{scrartcl}

%%%%%%%%%%%%%%%%%%%%%%%%%%%%%%%%%%%%%%%%%%%%%%%%%%%%%%%%%%%%%%%%%
%																%
%			Pakete werden definiert.							%
%																%
%%%%%%%%%%%%%%%%%%%%%%%%%%%%%%%%%%%%%%%%%%%%%%%%%%%%%%%%%%%%%%%%%

% Babel Sprachpaket, spezifish ngerman, nutzen
\usepackage[ngerman]{babel}
% Referenzen ermöglichen
\usepackage{hyperref}
\hypersetup{
	colorlinks=true,	% false: eingerahmte Referenzen; true: eingefärbte Referenzen
	linkcolor=blue,     % Farbe der internen Referenzen (falls eingerahmte Referenzen: linkbordercolor)
	citecolor=blue,     % Farbe der Quellen-Referenzen
	filecolor=blue,     % Farbe von Datei-Referenzen
	urlcolor=blue     	% Farbe von URL-Referenzen
}
% Einstellug um URL's umbrechen zu können. Mit \do\[Zeichen] werden Zeichen bestimmt, an denen eine URL umgebrochen werden darf.
\usepackage{etoolbox}
\apptocmd{\UrlBreaks}{\do\-\do\%\do\.}{}{}
% Schöne Referenzen von varioref nutzen, in deutsch
\usepackage[ngerman]{varioref}
% Verschiedene Symbole nutzbar machen
\usepackage{amsmath,amssymb,latexsym,amsfonts,amsthm,amsbsy,qtree}
% Url's vernünftig darstellen
\usepackage{url}
% Um Akronyme zu erlauben, aber nur wenn diese auch tatsächlich genutzt werden.
\usepackage[printonlyused]{acronym}
% Universal Character Set einbinden und damit deutsche Umlaute nutzbar machen
\usepackage[utf8]{inputenc} 
% Grafiken einbinden
\usepackage{graphicx}
% Optionen für die Figure Umgebung setzen (h= here, t= top, b= bottom, p= page(eigene Seite), != override(erwzingt die exakte Position)
\usepackage{float}
% Hübsche Header für die Abgabe
\usepackage{fancyhdr}
% Hauptsächlich für \midrule \toprule und \bottomrule in Tabellen
\usepackage{booktabs}
% Ermöglicht itemize-, enumerate- und description- Umegebungen anzupassen. (vgl. https://de.wikibooks.org/wiki/LaTeX-W%C3%B6rterbuch:_enumitem)
\usepackage{enumitem}
% Ermöglicht Einstellmöglichkeiten für captions über \captionsetup.
\usepackage[justification=centering]{caption}
% Zitierweise natbib zur Verfügung stellen.
\usepackage{natbib}
\bibliographystyle{plainnat}
% Einfügen von PDFs mit \include{file}
\usepackage{pdfpages}
\usepackage[official]{eurosym}

%%%%%%%%%%%%%%%%%%%%%%%%%%%%%%%%%%%%%%%%%%%%%%%%%%%%%%%%%%%%%%%%%
%																%
%			Daten eintragen.									%
%																%
%%%%%%%%%%%%%%%%%%%%%%%%%%%%%%%%%%%%%%%%%%%%%%%%%%%%%%%%%%%%%%%%%

% Die Autoren der Abgabe.
\newcommand{\teilnehmerI}{Tom Dombeck}
\newcommand{\mattI}{4510671}
\newcommand{\mailI}{todo@uni-bremen.de}

\newcommand{\teilnehmerII}{Lasse Warnke}
\newcommand{\mattII}{4515161}
\newcommand{\mailII}{lwarnke@uni-bremen.de}

% Gruppennummer[Tutoriumsbuchstaben+Gruppennummer]?
\newcommand{\thisgroup}{A01}
% Wann ist Abgabe?
\newcommand{\abgabedatum}{28.01.2019}
% Die wie vielte Abgabe ist es?
\newcommand{\nummer}{3}
% Was war Thema?
\newcommand{\thema}{Kosten-Nutzen-Betrachtung}
% Deine TutorIn?
\newcommand{\thistutor}{Tim Haß}

% Welches Semester?
\newcommand{\thissemester}{WiSe 2018/19}
% Welcher Kurs?
\newcommand{\thiscourse}{Wirtschaftsinformatik 1}
% Abkürzung für den Kurs?
\newcommand{\thisshortcourse}{WI1}

%%%%%%%%%%%%%%%%%%%%%%%%%%%%%%%%%%%%%%%%%%%%%%%%%%%%%%%%%%%%%%%%%
%																%
%			Automatisches Generieren							%
%																%
%%%%%%%%%%%%%%%%%%%%%%%%%%%%%%%%%%%%%%%%%%%%%%%%%%%%%%%%%%%%%%%%%

% Kopfzeile wird generiert
\pagestyle{fancy}
\fancyhead{} 													% Clear
	\fancyhead[LO,RE]{\thissemester \\ \thisshortcourse} 		% Kurs und Semester auf der linken Seite
	\fancyhead[RO,LE]{TutorIn: \thistutor \\ Gruppe: \thisgroup }% Tutor und Gruppennummer auf der rechten Seite
	\fancyfoot{} 												% Clear
	\cfoot{\thepage} 											% Seitenzahl am unteren Rand
	\setlength{\headsep}{2cm} 									% 2 cm Freiraum zwischen Header und Text

\begin{document}
\setlength{\parindent}{0em}

% Deckblatt wird generiert
\begin{titlepage}
	\vspace*{\baselineskip}			% Freiraum zwischen ersten Eintrag und oberer Seitenrand
	\centering						% Alles Nachfolgende zentrieren
	\LARGE							% Alles Nachfolgende groß
	\thiscourse \\ 					% Den Kurs einfügen
	\vspace{1cm}					% Skip 1 cm
	{\Huge 							% {} - Eingerahmte riesig
	\textbf{Abgabe \nummer: \thema}} \\ % Augabennummer und Thema hinzugefügt
	\vspace{1.5cm} 					% Skip 1.5 cm
	TutorIn: \thistutor \\ 			% TutorIn eingefügt
	\abgabedatum \\ 				% Abgabedatum eingefügt
	\vfill 							% Den Rest der Seite füllen
	Gruppe: \thisgroup \\ 			% Gruppennummer eingefügt
	\vspace{.5cm} 					% Skip 0.5 cm
	\large 							% Etwas kleinere Groß
	\begin{tabular}{c|c} 			% Daten der Studierenden werden eingefügt
	\teilnehmerI	& \teilnehmerII \\ %& \teilnehmerIII
	\mattI	& \mattII \\ %&  \mattIII
	\mailI	& \mailII \\ %& \mailIII
	\end{tabular} 
\end{titlepage}

\thispagestyle{empty}
\tableofcontents

\newpage
\setcounter{page}{1} % Setzt die Nummierung der Seite auf 1.

\section*{Nutzwertanalyse}
\addcontentsline{toc}{section}{Nutzwertanalyse}

\subsection*{Kriterien}
\addcontentsline{toc}{subsection}{Kriterien}

Im Folgenden sind die Kriterien der Nutzwertanalyse und die Gewichtung dieser in Tabellenform aufgelistet. Anschließend werden die Kriterien noch kurz beschrieben.

\begin{table}[h]
\begin{tabular}{|p{0,5cm}|p{8,5cm}|p{2,5cm}|p{2,6cm}|}
\hline
 & Kriterium & Gewichtung Kriterium & Gewichtung Unterkriterien\\
\hline
\textbf{1.} &  \textbf{Aufwand für die Einhaltung von Gesetzen und Verwaltungsvorschriften} & & \\
\hline
\hline
\textbf{2.} & \textbf{Auswirkung auf IT-Sicherheit} & & \\
\hline
2.1 & Auswirkung auf das Bedrohungspotential & & \\
\hline
2.2 & Auswirkung auf die Stabilität des Systems & & \\
\hline
\hline
\textbf{3.} & \textbf{Mitarbeiter bezogenen Wirkung} & & \\
\hline
3.1 & Zufriedenheit/Motivation/Attraktivität der Arbeitsbedingung & & \\
\hline
3.2 & Qualifikationssicherung & & \\
\hline
\hline
\textbf{4.} & \textbf{Studierenden bezogenen Wirkung} & & \\
\hline
\hline
\textbf{5.} & \textbf{Auswirkung auf die IT-Organisation} & & \\
\hline
5.1 & Integration von bestehenden Programmen & & \\
\hline
5.2 & Auswirkung auf die Komplexität der Systemumgebung & & \\
\hline
5.3 & Auswirkung auf das IT-Betriebsmanagement & & \\
\hline
\hline
\textbf{6.} & \textbf{Auswirkung auf externe Adressaten} & & \\
\hline
\hline
\textbf{7.} & \textbf{Produktivitätseffekte} & & \\
\hline
\hline
\hline
 & Gesamt & 100 & \\
\hline
\end{tabular}
\caption*{Bei den Kriterien und der Gewichtung dieser wurde sich sehr stark an dem Projekt der Landeshauptstadt München aus dem Jahr 2003 orientiert [Quelle].}
\end{table}
\vspace{0,25cm}

Das erste Kriterium misst den Aufwand, der betrieben werden muss, um alle gesetzlichen Vorschriften einzuhalten. Hierzu zählen auch alle Anforderungen bezüglich Datensicherheit und Datenschutz. Da sich im Netzwerk der Universität teilweise sehr sensible Daten wie Forschungsergebnisse oder persönliche Daten der Studierenden befinden, haben wir dieses Kriterium relativ schwer gewichtet.

Das zweite Kriterium misst die Auswirkungen auf die Sicherheit des IT-Systems. Ist das System angreifbarer von außen und wie stabil ist das System? Die Begründung für die schwere Gewichtung ist ähnlich wie bei Kriterium 1 (s.o.). 

Das dritte Kriterium beschreibt die Auswirkung auf die Mitarbeiter der Universität. Sie haben täglich mit den Client-Rechnern und den Fachanwendungen zu tun. Als besondere Punkte gibt es hier die Attraktivität des Arbeitsplatzes für alle Mitarbeiter und die Qualifikationssicherung dieser. Deswegen haben wir diesem Punkt mit \% eine relativ große Gewichtung zugesprochen. 

Das vierte Kriterium beschreibt hingegen die Auswirkung auf die Studierenden der Universität Bremen. Auch sie arbeiten an den Client-Rechnern der Universität, allerdings sind sie im Allgemeinen nicht so angewiesen und nutzen im Vergleich auch weniger Rechner. Deswegen liegt hier die Gewichtung tiefer als bei den Mitarbeitern. 

Das fünfte Kriterium misst die Auswirkung auf die IT-Organisation. Hierzu zählt die Integration der Fachanwendungen in das System, die Auswirkungen auf die Komplexität der Systemumgebung und die Auswirkung auf das IT-Betriebsmanagement. 

Das sechste Kriterium beschreibt die Auswirkung auf externe Adressaten, also die Kommunikation über die Grenzen der Universität hinaus. Dieses Kriterium ist nicht besonders schwer zu gewichten, da die Kommunikation größtenteils bei allen Systemen und Fachanwendungen gleich ablaufen kann.

Das siebte Kriterium misst die Produktivitätseffekte wie fehlende Standards bei Programmen, längere oder kürzere Bearbeitungszeiten, eventuelle Ausfälle wegen der Umstellung des Systems usw. Da sich diese Probleme, falls sie auftreten, wahrscheinlich auf die Zeit der Umstellung beschränken, ist dieses Kriterium relativ schwach gewichtet.

\subsection*{Durchführung Nutzwertanalyse}
\addcontentsline{toc}{subsection}{Durchführung Nutzwertanalyse}

\section*{Kostenabschätzung nach TCO-Modell}
\addcontentsline{toc}{section}{Kostenabschätzung nach TCO-Modell}

\subsection*{Kostenfaktoren}
\addcontentsline{toc}{subsection}{Kostenfaktoren}

\textbf{Lizenzen.} Der Faktor Lizenzen umfasst die Lizenzkosten für das neue Betriebssystem sowie die Lizenzkosten für die nicht mit Linux kompatiblen Fachanwendungen.
\newline

\textbf{Entwicklung und Anpassungen.} Diese umfassen die Kosten für die Anpassung oder im Notfall für dei Neuentwicklung von inkompatiblen Fachanwendeungen für welche es keine Alternativen oder zumindest keine Sinnvollen Alternativen unter Linux gibt.
\newline

\textbf{Schulungen.} Die Kosten für die Schulung der Mitarbeiter für die Nutzung des neuen Systems.
\newline

\textbf{Personal.} Die Personalkosten die während des Projekts bzw. während der Umstellung anfallen. Also die Personalkosten der Personen und Fachkräfte unmittelbar an der Umstellung beteiligt sind.
\newline

\textbf{Administration.} Die Kosten für die Administration des Gesamtsystems.
\newline

\textbf{Optimierung.} Unter Optimierung sind die Kosten zusammengefasst welche durch nachträgliche Veränderungen am System oder durch Optimierung im laufenden Betrieb anfallen.
\newline

\textbf{Ausfall durch Umstellung.} Dies bezeichnet den Gewinnverlust der während der wirklichen Umstellung, also die Zeit während das System nicht benutzbar ist, anfällt.
\newline

\textbf{Verringerte Produktivität.} Dieser Faktor beschreibt den Effekt der Umstellung auf die Produktivität der Mitarbeiter. Also die negativen Auswirkungen auf diese aufgrund der neuen Umgebung.
\newpage

\subsection*{Durchführung der Kostenabschätzung}
\addcontentsline{toc}{subsection}{Durchführung der Kostenabschätzung}

Die Kostenabschätzung bezieht sich auf einen Zeitraum von 3 Jahren. Hierfür werden die bereits in Aufgabe c genannten Kostenfaktoren berücksichtigt.

\begin{table}[h]
\begin{tabular}{|p{5cm}|p{8cm}|p{2cm}|}
\hline
Kostenfaktoren & Teilkosten & Gesamt \\
\hline
Lizenzen & Ubuntu an sich kostenlos. Bei nutzung von Ubuntu Advantage Lizenzkosten von 1.800.000\euro . \newline 70 Fachanwendungen: 30.000\euro \space pro Lizenz auf 3 Jahre & 2.100.000\euro \newline bzw. \newline 3.900.000\euro \\
\hline
Entwicklung und Anpassung & Anpassung / Neuentwicklung von 30 Fachanwendungen. 30.000\euro \space pro Anwendung & 900.000\euro \\
\hline
Schulungen & 1-Tägige Schulungen für 3500 Mitarbeiter. ca. 400\euro \space pro Person & 1.400.000\euro \\
\hline
Personalkosten & 120 Mitarbeiter. ca. 3500\euro \space pro Person pro Monat auf einen Zeitraum von 5 Monaten & 2.100.000\euro \\
\hline
Administration & Kein nennenswerter Unterschied zur aktuellen Situation (Windows). Möglicherweise sogar günstiger. & \\
\hline
Optimierung & Optimierung und Verbesserung. ca. 10.000\euro pro Monat & 360.000\euro \\
\hline
Ausfall durch Umstellung & & \\
\hline
Verringerte Produktivität & Verringerung der Produktivität von ca. 15\% für 3 Wochen bei einem duchschnittlichen Gehalt von ca. 3000\euro \space im Monat & 1.181.250\euro \\
\hline
Geasmt & & 8.041.250\euro \newline bzw. \newline 9.841.250\euro \space mit Ubuntu Advantage \\
\hline
\end{tabular}
\end{table}

\section*{Fazit und Empfehlung}
\addcontentsline{toc}{section}{Fazit und Empfehlung}

\newpage
\addcontentsline{toc}{section}{Literaturverzeichnis}
\bibliography{Literaturdatenbank}

\end{document}
