% Createt 2016 by Jens Wiebe for the use in the course: Wirtschaftinformatik 1 at Universität Bremen
% The use of TexLive 2016 as compiler-package and TexStudio 2.11.2 as editor are recommended

\documentclass[12pt,utf8]{scrartcl}

%%%%%%%%%%%%%%%%%%%%%%%%%%%%%%%%%%%%%%%%%%%%%%%%%%%%%%%%%%%%%%%%%
%																%
%			Pakete werden definiert.							%
%																%
%%%%%%%%%%%%%%%%%%%%%%%%%%%%%%%%%%%%%%%%%%%%%%%%%%%%%%%%%%%%%%%%%

% Babel Sprachpaket, spezifish ngerman, nutzen
\usepackage[ngerman]{babel}
% Referenzen ermöglichen
\usepackage{hyperref}
\hypersetup{
	colorlinks=true,	% false: eingerahmte Referenzen; true: eingefärbte Referenzen
	linkcolor=blue,     % Farbe der internen Referenzen (falls eingerahmte Referenzen: linkbordercolor)
	citecolor=blue,     % Farbe der Quellen-Referenzen
	filecolor=blue,     % Farbe von Datei-Referenzen
	urlcolor=blue     	% Farbe von URL-Referenzen
}
% Einstellug um URL's umbrechen zu können. Mit \do\[Zeichen] werden Zeichen bestimmt, an denen eine URL umgebrochen werden darf.
\usepackage{etoolbox}
\apptocmd{\UrlBreaks}{\do\-\do\%\do\.}{}{}
% Schöne Referenzen von varioref nutzen, in deutsch
\usepackage[ngerman]{varioref}
% Verschiedene Symbole nutzbar machen
\usepackage{amsmath,amssymb,latexsym,amsfonts,amsthm,amsbsy,qtree}
% Url's vernünftig darstellen
\usepackage{url}
% Um Akronyme zu erlauben, aber nur wenn diese auch tatsächlich genutzt werden.
\usepackage[printonlyused]{acronym}
% Universal Character Set einbinden und damit deutsche Umlaute nutzbar machen
\usepackage[utf8]{inputenc}
% Grafiken einbinden
\usepackage{graphicx}
% Optionen für die Figure Umgebung setzen (h= here, t= top, b= bottom, p= page(eigene Seite), != override(erwzingt die exakte Position)
\usepackage{float}
% Hübsche Header für die Abgabe
\usepackage{fancyhdr}
% Hauptsächlich für \midrule \toprule und \bottomrule in Tabellen
\usepackage{booktabs}
% Ermöglicht itemize-, enumerate- und description- Umegebungen anzupassen. (vgl. https://de.wikibooks.org/wiki/LaTeX-W%C3%B6rterbuch:_enumitem)
\usepackage{enumitem}
% Ermöglicht Einstellmöglichkeiten für captions über \captionsetup.
\usepackage[justification=centering]{caption}
% Zitierweise natbib zur Verfügung stellen.
\usepackage{natbib}
\bibliographystyle{plainnat}
% Einfügen von PDFs mit \include{file}
\usepackage{pdfpages}
\usepackage[official]{eurosym}

%%%%%%%%%%%%%%%%%%%%%%%%%%%%%%%%%%%%%%%%%%%%%%%%%%%%%%%%%%%%%%%%%
%																%
%			Daten eintragen.									%
%																%
%%%%%%%%%%%%%%%%%%%%%%%%%%%%%%%%%%%%%%%%%%%%%%%%%%%%%%%%%%%%%%%%%

% Die Autoren der Abgabe.
\newcommand{\teilnehmerI}{Tom Dombeck}
\newcommand{\mattI}{4510671}
\newcommand{\mailI}{todo@uni-bremen.de}

\newcommand{\teilnehmerII}{Lasse Warnke}
\newcommand{\mattII}{4515161}
\newcommand{\mailII}{lwarnke@uni-bremen.de}

% Gruppennummer[Tutoriumsbuchstaben+Gruppennummer]?
\newcommand{\thisgroup}{A01}
% Wann ist Abgabe?
\newcommand{\abgabedatum}{28.01.2019}
% Die wie vielte Abgabe ist es?
\newcommand{\nummer}{3}
% Was war Thema?
\newcommand{\thema}{Kosten-Nutzen-Betrachtung}
% Deine TutorIn?
\newcommand{\thistutor}{Tim Haß}

% Welches Semester?
\newcommand{\thissemester}{WiSe 2018/19}
% Welcher Kurs?
\newcommand{\thiscourse}{Wirtschaftsinformatik 1}
% Abkürzung für den Kurs?
\newcommand{\thisshortcourse}{WI1}

%%%%%%%%%%%%%%%%%%%%%%%%%%%%%%%%%%%%%%%%%%%%%%%%%%%%%%%%%%%%%%%%%
%																%
%			Automatisches Generieren							%
%																%
%%%%%%%%%%%%%%%%%%%%%%%%%%%%%%%%%%%%%%%%%%%%%%%%%%%%%%%%%%%%%%%%%

% Kopfzeile wird generiert
\pagestyle{fancy}
\fancyhead{} 													% Clear
	\fancyhead[LO,RE]{\thissemester \\ \thisshortcourse} 		% Kurs und Semester auf der linken Seite
	\fancyhead[RO,LE]{TutorIn: \thistutor \\ Gruppe: \thisgroup }% Tutor und Gruppennummer auf der rechten Seite
	\fancyfoot{} 												% Clear
	\cfoot{\thepage} 											% Seitenzahl am unteren Rand
	\setlength{\headsep}{2cm} 									% 2 cm Freiraum zwischen Header und Text

\begin{document}
\setlength{\parindent}{0em}

% Deckblatt wird generiert
\begin{titlepage}
	\vspace*{\baselineskip}			% Freiraum zwischen ersten Eintrag und oberer Seitenrand
	\centering						% Alles Nachfolgende zentrieren
	\LARGE							% Alles Nachfolgende groß
	\thiscourse \\ 					% Den Kurs einfügen
	\vspace{1cm}					% Skip 1 cm
	{\Huge 							% {} - Eingerahmte riesig
	\textbf{Abgabe \nummer: \thema}} \\ % Augabennummer und Thema hinzugefügt
	\vspace{1.5cm} 					% Skip 1.5 cm
	TutorIn: \thistutor \\ 			% TutorIn eingefügt
	\abgabedatum \\ 				% Abgabedatum eingefügt
	\vfill 							% Den Rest der Seite füllen
	Gruppe: \thisgroup \\ 			% Gruppennummer eingefügt
	\vspace{.5cm} 					% Skip 0.5 cm
	\large 							% Etwas kleinere Groß
	\begin{tabular}{c|c} 			% Daten der Studierenden werden eingefügt
	\teilnehmerI	& \teilnehmerII \\ %& \teilnehmerIII
	\mattI	& \mattII \\ %&  \mattIII
	\mailI	& \mailII \\ %& \mailIII
	\end{tabular}
\end{titlepage}

\thispagestyle{empty}
\tableofcontents

\newpage
\setcounter{page}{1} % Setzt die Nummierung der Seite auf 1.

\section*{Nutzwertanalyse}
\addcontentsline{toc}{section}{Nutzwertanalyse}

\subsection*{Kriterien}
\addcontentsline{toc}{subsection}{Kriterien}

Im Folgenden sind die Kriterien der Nutzwertanalyse und die Gewichtung dieser in Tabellenform aufgelistet. Anschließend werden die Kriterien noch kurz beschrieben.

\begin{table}[h]
\begin{tabular}{|p{0,5cm}|p{12cm}|p{2cm}|}
\hline
 & Kriterien & Gewichtung Kriterien\\
\hline
\textbf{1.} &  \textbf{Einhaltung von Gesetzen und Verwaltungsvorschriften}  & 26,2\\
\hline
\hline
\textbf{2.} & \textbf{IT-Sicherheit} & 26,2\\
\hline
2.1 & Bedrohungspotential & 15,7 \\
\hline
2.2 & Stabilität des Systems & 10,5 \\
\hline
\hline
\textbf{3.} & \textbf{Mitarbeiter bezogenen Wirkung} & 16,7\\
\hline
3.1 & Attraktivität der Arbeitsbedingung & 10,0 \\
\hline
3.2 & Qualifikationssicherung & 6,7 \\
\hline
\hline
\textbf{4.} & \textbf{Studierenden bezogenen Wirkung} & 9,4\\
\hline
\hline
\textbf{5.} & \textbf{IT-Organisation} & 16,7\\
\hline
5.1 & Integration von bestehenden Programmen & 3,3 \\
\hline
5.2 & Komplexität der Systemumgebung & 5,0 \\
\hline
5.3 & IT-Betriebsmanagement & 8,4 \\
\hline
\hline
\textbf{6.} & \textbf{Externe Adressaten} & 2,4\\
\hline
\hline
\textbf{7.} & \textbf{Produktivitätseffekte} & 2,4\\
\hline
\hline
\hline
 & Gesamt & 100\\
\hline
\end{tabular}
\caption*{Bei den Kriterien und der Gewichtung dieser wurde sich sehr stark an dem Projekt der Landeshauptstadt München aus dem Jahr 2003 orientiert [Quelle].}
\end{table}

\begin{table}[h!]
\centering
\begin{tabular}{|r|r|r|r|r|r|r|r|c|l|}
\hline
Bewertungskriterien & 1 & 2 & 3 & 4 & 5 & 6 & 7 & Gewicht & Faktor \\
\hline
1 & X & 1 & 2 & 2 & 2 & 2 & 2 & 11 & 0,262 \\
\hline
2 & 1 & X & 2 & 2 & 2 & 2 & 2 & 11 & 0,262 \\
\hline
3 & 0 & 0 & X & 2 & 1 & 2 & 2 & 7 & 0,167 \\
\hline
4 & 0 & 0 & 0 & X & 0 & 2 & 2 & 4 & 0,094 \\
\hline
5 & 0 & 0 & 1 & 2 & X & 2 & 2 & 7 & 0,167 \\
\hline
6 & 0 & 0 & 0 & 0 & 0 & X & 1 & 1 & 0,024 \\
\hline
7 & 0 & 0 & 0 & 0 & 0 & 1 & X & 1 & 0,024 \\
\hline
Summe: & & & & & & & & 42 & 1 \\
\hline
\end{tabular}
\caption*{Die Berechnung der Gewichtung der Kriterien}
\end{table}

\newpage
\textbf{Einhaltung von Gesetzen und Verwaltungsvorschriften.} Das erste Kriterium misst den Aufwand, der betrieben werden muss, um alle gesetzlichen Vorschriften einzuhalten. Hierzu zählen auch alle Anforderungen bezüglich Datensicherheit und Datenschutz. Da sich im Netzwerk der Universität teilweise sehr sensible Daten wie Forschungsergebnisse oder persönliche Daten der Studierenden befinden, haben wir dieses Kriterium relativ schwer gewichtet.
\newline

\textbf{IT-Sicherheit.} Das zweite Kriterium misst die Auswirkungen auf die Sicherheit des IT-Systems. Ist das System angreifbarer von außen und wie stabil ist das System? Die Begründung für die schwere Gewichtung ist ähnlich wie bei Kriterium 1 (s.o.). 
\newline

\textbf{Mitarbeiter bezogenen Wirkung.} Das dritte Kriterium beschreibt die Auswirkung auf die Mitarbeiter der Universität. Sie haben täglich mit den Client-Rechnern und den Fachanwendungen zu tun. Als besondere Punkte gibt es hier die Attraktivität des Arbeitsplatzes für alle Mitarbeiter und die Qualifikationssicherung dieser. Deswegen haben wir diesem Punkt mit 16,7 \% eine relativ große Gewichtung zugesprochen. 
\newline

\textbf{Studierenden bezogenen Wirkung.} Das vierte Kriterium beschreibt hingegen die Auswirkung auf die Studierenden der Universität Bremen. Auch sie arbeiten an den Client-Rechnern der Universität, allerdings sind sie im Allgemeinen nicht so angewiesen und nutzen im Vergleich auch weniger Rechner. Deswegen liegt hier die Gewichtung tiefer als bei den Mitarbeitern. 
\newline

\textbf{IT-Organisation.} Das fünfte Kriterium misst die Auswirkung auf die IT-Organisation. Hierzu zählt die Integration der Fachanwendungen in das System, die Auswirkungen auf die Komplexität der Systemumgebung und die Auswirkung auf das IT-Betriebsmanagement. Dieser Punkt ist ebenfalls relativ schwer gewichtet.
\newline

\textbf{Externe Adressaten.} Das sechste Kriterium beschreibt die Auswirkung auf externe Adressaten, also die Kommunikation über die Grenzen der Universität hinaus. Dieses Kriterium ist nicht besonders schwer zu gewichten, da die Kommunikation größtenteils bei allen Systemen und Fachanwendungen gleich ablaufen kann. 
\newline

\textbf{Produktivitätseffekte.} Das siebte Kriterium misst die Produktivitätseffekte wie fehlende Standards bei Programmen, längere oder kürzere Bearbeitungszeiten, eventuelle Ausfälle wegen der Umstellung des Systems usw. Da sich diese Probleme, falls sie auftreten, wahrscheinlich auf die Zeit der Umstellung beschränken, ist dieses Kriterium relativ schwach gewichtet. 

\newpage
\subsection*{Durchführung Nutzwertanalyse}
\addcontentsline{toc}{subsection}{Durchführung Nutzwertanalyse}

Wir haben uns für eine Skaleneinteilung von 1 (sehr schlecht) bis 10 (sehr gut) entschieden. 

\begin{table}[h]
\centering
\begin{tabular}{|p{5cm}|p{1cm}|p{2,2cm}|p{2cm}|p{2,2cm}|p{2cm}|}
\hline
Kriterien & & \textbf{Linux/OSS} & & \textbf{Windows OS/Office} & \\
 & Faktor & Zielerfüllung & Nutzwert & Zielerfüllung & Nutzwert \\
\hline
1. Einhaltung von Gesetzen und Verwaltungsvorschriften & 0,262 & 5 & 1,310 & 3 & 0,786 \\
\hline
2. IT-Sicherheit & 0,262 & 9 & 2,358 & 6 & 1,572 \\
\hline
2.1 Bedrohungspotential & 0,157 & 8 & 1,256 & 5 & 0,785 \\
\hline
2.2 Stabilität des Systems & 0,105 & 10 & 1,050 & 7 & 0,735 \\
\hline
3. Mitarbeiter bezogenen Wirkung & 0,167 & 6 & 1,002 & 9 & 1,503 \\
\hline
3.1 Attraktivität der Arbeitsbedingung & 0,100 & 6 & 0,600 & 9 & 0,900 \\
\hline
3.2 Qualifikationssicherung & 0,067 & 5 & 0,335 & 8 & 0,536 \\
\hline
4. Studierenden bezogenen Wirkung & 0,094 & 7 & 0,658 & 9 & 0,846 \\
\hline
5. IT-Organisation & 0,167 & 7 & 1,169 & 6 & 1,002 \\
\hline
5.1 Integration von bestehenden Anwendungen & 0,033 & 5 & 0,165 & 10 & 0,330 \\
\hline
5.2 Komplexität der Systemumgebung & 0,050 & 9 & 0,450 & 7 & 0,350 \\
\hline
5.3 IT-Betriebsmanagement & 0,084 & 8 & 0,672 & 4 & 0,336 \\
\hline
6. Externe Adressaten & 0,024 & 8 & 0,192 & 9 & 0,216 \\
\hline
7. Produktivitätseffekte & 0,024 & 6 & 0,144 & 9 & 0,216 \\
\hline
 & & Summe: & 11,361 & Summe: & 10,113 \\
\hline
\end{tabular}
\caption*{Die Nutzwertanalyse}
\end{table}

\newpage
\section*{Kostenabschätzung nach TCO-Modell}
\addcontentsline{toc}{section}{Kostenabschätzung nach TCO-Modell}

\subsection*{Kostenfaktoren}
\addcontentsline{toc}{subsection}{Kostenfaktoren}

\textbf{Lizenzen.} Der Faktor Lizenzen umfasst die Lizenzkosten für das neue Betriebssystem sowie die Lizenzkosten für die nicht mit Linux kompatiblen Fachanwendungen.
\newline

\textbf{Entwicklung und Anpassungen.} Diese umfassen die Kosten für die Anpassung oder im Notfall für dei Neuentwicklung von inkompatiblen Fachanwendungen für welche es keine Alternativen oder zumindest keine Sinnvollen Alternativen unter Linux gibt.
\newline

\textbf{Schulungen.} Die Kosten für die Schulung der Mitarbeiter für die Nutzung des neuen Systems.
\newline

\textbf{Personal.} Die Personalkosten die während des Projekts bzw. während der Umstellung anfallen. Also die Personalkosten der Personen und Fachkräfte unmittelbar an der Umstellung beteiligt sind.
\newline

\textbf{Administration.} Die Kosten für die Administration des Gesamtsystems.
\newline

\textbf{Optimierung.} Unter Optimierung sind die Kosten zusammengefasst welche durch nachträgliche Veränderungen am System oder durch Optimierung im laufenden Betrieb anfallen.
\newline

\textbf{Ausfall durch Umstellung.} Dies bezeichnet den Gewinnverlust der während der wirklichen Umstellung, also die Zeit während das System nicht benutzbar ist, anfällt.
\newline

\textbf{Verringerte Produktivität.} Dieser Faktor beschreibt den Effekt der Umstellung auf die Produktivität der Mitarbeiter. Also die negativen Auswirkungen auf diese aufgrund der neuen Umgebung.
\newpage

\subsection*{Durchführung der Kostenabschätzung}
\addcontentsline{toc}{subsection}{Durchführung der Kostenabschätzung}

Die Kostenabschätzung bezieht sich auf einen Zeitraum von 3 Jahren. Hierfür werden die bereits in Aufgabe c genannten Kostenfaktoren berücksichtigt.

\begin{table}[h]
\begin{tabular}{|p{5cm}|p{8cm}|p{2cm}|}
\hline
Kostenfaktoren & Teilkosten & Gesamt \\
\hline
Lizenzen & Ubuntu an sich kostenlos. Bei Nutzung von Ubuntu Advantage Lizenzkosten von 1.800.000\euro . \newline 70 Fachanwendungen: 30.000\euro \space pro Lizenz auf 3 Jahre & 2.100.000\euro \newline bzw. \newline 3.900.000\euro \\
\hline
Entwicklung und Anpassung & Anpassung / Neuentwicklung von 30 Fachanwendungen. 30.000\euro \space pro Anwendung & 900.000\euro \\
\hline
Schulungen & 1-Tägige Schulungen für 3500 Mitarbeiter. ca. 400\euro \space pro Person & 1.400.000\euro \\
\hline
Personalkosten & 120 Mitarbeiter. ca. 3500\euro \space pro Person pro Monat auf einen Zeitraum von 5 Monaten & 2.100.000\euro \\
\hline
Administration & Kein nennenswerter Unterschied zur aktuellen Situation (Windows). Möglicherweise sogar günstiger. & \\
\hline
Optimierung & Optimierung und Verbesserung. ca. 10.000\euro pro Monat & 360.000\euro \\
\hline
Ausfall durch Umstellung & 20\% Ausfall bzw. Zeit in welcher das System nicht genutzt werden kann in der Woche der Umstellung & 525.000\euro \\
\hline
Verringerte Produktivität & Verringerung der Produktivität von ca. 15\% für 3 Wochen bei einem durchschnittlichen Gehalt von ca. 3000\euro \space im Monat & 1.181.250\euro \\
\hline
Gesamt & & 8.566.250\euro \newline bzw. \newline 10.366.250\euro \space mit Ubuntu Advantage \\
\hline
\end{tabular}
\end{table}

Der Faktor Lizenzkosten teilt sich in zwei Bereiche, einmal die Kosten für das Betriebssystem und andererseits die für die nicht mit Linux kompatiblen Fachanwendungen (also die Anwendungen die neu angeschafft werden müssen).

Das von uns gewählte Linux-Betriebssystem Ubuntu ist an sich erstmal kostenlos, auch im kommerziellen Bereich. Allerdings bietet Canonical (der Distributor von Ubuntu) eine für Unternehmen gedachte Version mit dem Namen \emph{Ubuntu Advantage} welche erweiterten Support bietet. Diese hat einen Preis von 150\euro \space pro Lizenz und Jahr (eine Lizenz ist für ein Gerät gültig). Da es für uns nicht wirklich ersichtlich ist ob die erweiterte Version von Ubuntu bei einer großen Anzahl von Systemen sinnvoll ist führen wir die Kosten für beide Möglichkeiten auf.

Bei den Lizenzkosten für die inkompatiblen Fachanwendungen ist es schwierig die genauen Kosten festzulegen da es keine Angabe dazu gibt um welche Anwendungen es sich genau handelt. Wir gehen in unserer Abschätzung davon aus, dass es für 70 von den inkompatiblen Anwendungen alternativen für Linux gibt. Da es keine genaueren Angaben zu den eigentlichen Anwendungen gibt legen wir ein Preis von durchschnittlich 10.000\euro \space pro Anwendung pro Jahr fest.

Die restlichen 30 Fachanwendungen müssen entweder an Linux angepasst werden oder teilweise neu entwickelt werden. Da es zu den Kosten von solchen Anpassungen keine wirklichen Quellen gibt gehen wir von Kosten von ca. 30.000\euro pro Anwendung aus.

Einen großen Teil der Kosten muss für die Schulung bzw. Einführung der Mitarbeiter in die neue Arbeitsumgebung aufgewendet werden. Bei den meisten "Anfänger-"Schulungen für Linux gehen über 2 bis 3 Tage und kosten zwischen 1200\euro \space und 1500\euro \space pro Person . Wir haben uns für eine eintägige Schulung entschieden. Da es hierfür keine Angaben für die Kosten gibt haben wir im Vergleich zu den mehrtägigen Schulungen einen Preis von 400\euro \space pro Person festgelegt.

Die Personalkosten Umfassen die Gehälter der 120 Mitarbeiter die direkt an der Umstellung beteiligt sind. Da es keine Angabe über diese 120 Personen und z.B. deren Qualifikation, haben wir einen Durchschnittsgehalt von 3.500\euro \space festgelegt. In unserer Abschätzung dauert die Zeit in welcher diese Personen mit der Umstellung beschäftigt sind ca. 5 Monate.

Die Administration des neuen Systems dürfte im Vergleich zum Jetzigen keinen Mehraufwand machen und daher auch keine Mehrkosten bedeuten.



\newpage
\section*{Fazit und Empfehlung}
\addcontentsline{toc}{section}{Fazit und Empfehlung}

\newpage
\addcontentsline{toc}{section}{Literaturverzeichnis}
\bibliography{Literaturdatenbank}

\end{document}
