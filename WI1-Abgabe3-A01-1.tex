% Createt 2016 by Jens Wiebe for the use in the course: Wirtschaftinformatik 1 at Universität Bremen
% The use of TexLive 2016 as compiler-package and TexStudio 2.11.2 as editor are recommended

\documentclass[12pt,utf8]{scrartcl}

%%%%%%%%%%%%%%%%%%%%%%%%%%%%%%%%%%%%%%%%%%%%%%%%%%%%%%%%%%%%%%%%%
%																%
%			Pakete werden definiert.							%
%																%
%%%%%%%%%%%%%%%%%%%%%%%%%%%%%%%%%%%%%%%%%%%%%%%%%%%%%%%%%%%%%%%%%

% Babel Sprachpaket, spezifish ngerman, nutzen
\usepackage[ngerman]{babel}
% Referenzen ermöglichen
\usepackage{hyperref}
\hypersetup{
	colorlinks=true,	% false: eingerahmte Referenzen; true: eingefärbte Referenzen
	linkcolor=blue,     % Farbe der internen Referenzen (falls eingerahmte Referenzen: linkbordercolor)
	citecolor=blue,     % Farbe der Quellen-Referenzen
	filecolor=blue,     % Farbe von Datei-Referenzen
	urlcolor=blue     	% Farbe von URL-Referenzen
}
% Einstellug um URL's umbrechen zu können. Mit \do\[Zeichen] werden Zeichen bestimmt, an denen eine URL umgebrochen werden darf.
\usepackage{etoolbox}
\apptocmd{\UrlBreaks}{\do\-\do\%\do\.}{}{}
% Schöne Referenzen von varioref nutzen, in deutsch
\usepackage[ngerman]{varioref}
% Verschiedene Symbole nutzbar machen
\usepackage{amsmath,amssymb,latexsym,amsfonts,amsthm,amsbsy,qtree}
% Url's vernünftig darstellen
\usepackage{url}
% Um Akronyme zu erlauben, aber nur wenn diese auch tatsächlich genutzt werden.
\usepackage[printonlyused]{acronym}
% Universal Character Set einbinden und damit deutsche Umlaute nutzbar machen
\usepackage[utf8]{inputenc} 
% Grafiken einbinden
\usepackage{graphicx}
% Optionen für die Figure Umgebung setzen (h= here, t= top, b= bottom, p= page(eigene Seite), != override(erwzingt die exakte Position)
\usepackage{float}
% Hübsche Header für die Abgabe
\usepackage{fancyhdr}
% Hauptsächlich für \midrule \toprule und \bottomrule in Tabellen
\usepackage{booktabs}
% Ermöglicht itemize-, enumerate- und description- Umegebungen anzupassen. (vgl. https://de.wikibooks.org/wiki/LaTeX-W%C3%B6rterbuch:_enumitem)
\usepackage{enumitem}
% Ermöglicht Einstellmöglichkeiten für captions über \captionsetup.
\usepackage[justification=centering]{caption}
% Zitierweise natbib zur Verfügung stellen.
\usepackage{natbib}
\bibliographystyle{plainnat}
% Einfügen von PDFs mit \include{file}
\usepackage{pdfpages}

%%%%%%%%%%%%%%%%%%%%%%%%%%%%%%%%%%%%%%%%%%%%%%%%%%%%%%%%%%%%%%%%%
%																%
%			Daten eintragen.									%
%																%
%%%%%%%%%%%%%%%%%%%%%%%%%%%%%%%%%%%%%%%%%%%%%%%%%%%%%%%%%%%%%%%%%

% Die Autoren der Abgabe.
\newcommand{\teilnehmerI}{Tom Dombeck}
\newcommand{\mattI}{4510671}
\newcommand{\mailI}{todo@uni-bremen.de}

\newcommand{\teilnehmerII}{Lasse Warnke}
\newcommand{\mattII}{4515161}
\newcommand{\mailII}{lwarnke@uni-bremen.de}

% Gruppennummer[Tutoriumsbuchstaben+Gruppennummer]?
\newcommand{\thisgroup}{A01}
% Wann ist Abgabe?
\newcommand{\abgabedatum}{28.01.2019}
% Die wie vielte Abgabe ist es?
\newcommand{\nummer}{3}
% Was war Thema?
\newcommand{\thema}{Kosten-Nutzen-Betrachtung}
% Deine TutorIn?
\newcommand{\thistutor}{Tim Haß}

% Welches Semester?
\newcommand{\thissemester}{WiSe 2018/19}
% Welcher Kurs?
\newcommand{\thiscourse}{Wirtschaftsinformatik 1}
% Abkürzung für den Kurs?
\newcommand{\thisshortcourse}{WI1}

%%%%%%%%%%%%%%%%%%%%%%%%%%%%%%%%%%%%%%%%%%%%%%%%%%%%%%%%%%%%%%%%%
%																%
%			Automatisches Generieren							%
%																%
%%%%%%%%%%%%%%%%%%%%%%%%%%%%%%%%%%%%%%%%%%%%%%%%%%%%%%%%%%%%%%%%%

% Kopfzeile wird generiert
\pagestyle{fancy}
\fancyhead{} 													% Clear
	\fancyhead[LO,RE]{\thissemester \\ \thisshortcourse} 		% Kurs und Semester auf der linken Seite
	\fancyhead[RO,LE]{TutorIn: \thistutor \\ Gruppe: \thisgroup }% Tutor und Gruppennummer auf der rechten Seite
	\fancyfoot{} 												% Clear
	\cfoot{\thepage} 											% Seitenzahl am unteren Rand
	\setlength{\headsep}{2cm} 									% 2 cm Freiraum zwischen Header und Text

\begin{document}


% Deckblatt wird generiert
\begin{titlepage}
	\vspace*{\baselineskip}			% Freiraum zwischen ersten Eintrag und oberer Seitenrand
	\centering						% Alles Nachfolgende zentrieren
	\LARGE							% Alles Nachfolgende groß
	\thiscourse \\ 					% Den Kurs einfügen
	\vspace{1cm}					% Skip 1 cm
	{\Huge 							% {} - Eingerahmte riesig
	\textbf{Abgabe \nummer: \thema}} \\ % Augabennummer und Thema hinzugefügt
	\vspace{1.5cm} 					% Skip 1.5 cm
	TutorIn: \thistutor \\ 			% TutorIn eingefügt
	\abgabedatum \\ 				% Abgabedatum eingefügt
	\vfill 							% Den Rest der Seite füllen
	Gruppe: \thisgroup \\ 			% Gruppennummer eingefügt
	\vspace{.5cm} 					% Skip 0.5 cm
	\large 							% Etwas kleinere Groß
	\begin{tabular}{c|c} 			% Daten der Studierenden werden eingefügt
	\teilnehmerI	& \teilnehmerII \\ %& \teilnehmerIII
	\mattI	& \mattII \\ %&  \mattIII
	\mailI	& \mailII \\ %& \mailIII
	\end{tabular} 
\end{titlepage}

\thispagestyle{empty}
\tableofcontents

\newpage
\setcounter{page}{1} % Setzt die Nummierung der Seite auf 1.

\section{Nutzwertanalyse}

\subsection{Kriterien}
1. Aufwand für die Einhaltung von Gesetzen und Verwaltungsvorschriften

2. Auswirkung auf IT-Sicherheit

3. Mitarbeiter bezogenen Wirkung

Zufriedenheit/Motivation

4. Auswirkung auf die IT-Organisation

5. Auswirkung auf externe Adressaten

6. Erfüllung weiterer strategischer Punkte

\subsection{Durchführung Nutzwertanalyse}

\section{Kostenabschätzung nach TCO-Modell}

\subsection{Kriterien}

\subsection{Durchführung der Kostenabschätzung}

\section{Fazit und Empfehlung}

\newpage
\addcontentsline{toc}{section}{Literaturverzeichniss}%manual entry into table of content
\bibliography{Literaturdatenbank}

\end{document}
